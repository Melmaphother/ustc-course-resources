\documentclass[a4paper,12pt]{article}

\title{Data Privacy Homework 3}
\author{Daoyu Wang  PB21030794}
\date{January 1}
\usepackage{color}
\usepackage{graphicx}
\usepackage{amsmath}
\usepackage{algorithm}
\usepackage{algorithmicx}
\usepackage{indentfirst}
\usepackage{algpseudocode}
\usepackage{fancyhdr} 
\pagestyle{fancy}
% 页眉设置
\fancyhead[L]{Data Privacy Homework 3}
\fancyhead[R]{Daoyu Wang}
% \fancyhead[C]{}
% 页脚设置
% \fancyfoot[L]{left root}
% \fancyfoot[C]{\thepage}
% \fancyfoot[R]{right root}
\setlength{\headheight}{14.49998pt}
\addtolength{\topmargin}{-2.49998pt}

\begin{document}
\maketitle
\pagenumbering{roman}
\tableofcontents
\newpage
\section{Permutation Cipher}
\subsection{(a)}
Just need to solve the \textbf{inverse permutation} of the mapping $x$ to $\pi(x)$.
\begin{table}[!ht]
    \centering
    \begin{tabular}{|l|l|l|l|l|l|l|l|l|}
        \hline
        \textbf{x}             & \textbf{1} & \textbf{2} & \textbf{3} & \textbf{4} & \textbf{5} & \textbf{6} & \textbf{7} & \textbf{8} \\ \hline
        \textbf{$\pi^{-1}(x)$} & 2          & 4          & 6          & 1          & 8          & 3          & 5          & 7          \\ \hline
    \end{tabular}
\end{table}
\subsection{(b)}
We can devide the ciphertext into blocks of length $8$ and then use mapping $\pi^{-1}(x) \sim x $ to decrypt each block as follows:

\begin{equation}
    \pi^{-1}
    \begin{pmatrix}
        T & G & E & E & M & N & E & L \\
        1 & 2 & 3 & 4 & 5 & 6 & 7 & 8 \\
        N & N & T & D & R & O & E & O \\
        1 & 2 & 3 & 4 & 5 & 6 & 7 & 8 \\
        A & A & H & D & O & E & T & C \\
        1 & 2 & 3 & 4 & 5 & 6 & 7 & 8 \\
        S & H & A & E & I & R & L & M \\
        1 & 2 & 3 & 4 & 5 & 6 & 7 & 8 \\
    \end{pmatrix}
    =
    \begin{pmatrix}
        G & E & N & T & L & E & M & E \\
        2 & 4 & 6 & 1 & 8 & 3 & 5 & 7 \\
        N & D & O & N & O & T & R & E \\
        2 & 4 & 6 & 1 & 8 & 3 & 5 & 7 \\
        A & D & E & A & C & H & O & T \\
        2 & 4 & 6 & 1 & 8 & 3 & 5 & 7 \\
        H & E & R & S & M & A & I & L \\
        2 & 4 & 6 & 1 & 8 & 3 & 5 & 7 \\
    \end{pmatrix}
\end{equation}
which can be writed as:
\begin{equation}
    \mathrm{GENTLEMENDONOTREADEACHOTHERSMAIL}
\end{equation}
or
\begin{equation}
    \mathrm{GENTLEMEN~DO~NOT~READ~EACH~OTHERS~MAIL}.
\end{equation}
\section{Perfect Secrecy}
\subsection{(a)}
A cryptosystem has a perfect secrecy if
\begin{equation}
    \forall m \in M, c\in C, \Pr[M=m] = \Pr[M=m | C=c]
\end{equation}
which can be explained as the ciphertext $c$ does not give any information about the plaintext $m$.

Based on \textbf{Bayes' theorem}, we have:
\begin{equation}
    \Pr[M=m | C=c] = \frac{\Pr[C=c|M=m]\Pr[M=m]}{\Pr[C=c]}
\end{equation}

Since each key is chosen uniformly at random, so knowing $j$, there is only one key that encrypts $j$ to a $L(i, j)$ among the $n$ keys (Each number appears once in a column). Concerning $\Pr(c)$, each $L(i, j)$ appears $n$ times in the square among the $n^2$ possible cases. So $\Pr[C=c | M=m] = \frac{1}{n}$ and $\Pr[C=c] = \frac{n}{n^2}$. Thus, we have:
\begin{equation}
    \Pr[M=m | C=c] = \frac{\Pr[M=m]\cdot \frac{1}{n}}{\frac{n}{n^2}} = \Pr[M=m]
\end{equation}

In conclusion, the Latin Square Cryptosystem achieves perfect secrecy if the key is chosen uniformly at random.
\subsection{(b)}
Since the Latin Square Cryptosystem achieves perfect secrecy, we have:
\begin{equation}
    \Pr[M=m] = \Pr[M=m | C=c]
\end{equation}

So, we can deduce with the \textbf{Bayes' theorem} again:
\begin{equation}
    \begin{aligned}
        \forall c \in C, \Pr[C=c] & = \frac{\Pr[M = m]\cdot \Pr[C=c | M=m]}{\Pr[M=m | C=c]} \\
                                  & = \frac{\Pr[M = m]\cdot \Pr[C=c | M=m]}{\Pr[M=m]}       \\
                                  & = \Pr[C=c | M=m]                                        \\
    \end{aligned}
\end{equation}

As $|M| = |C| = |K|$, we know that there is only one key among $n$ that encrypts $m$ to $c$. So
\begin{equation}
    \forall c\in C, \Pr[C=c | M=m] = \frac{1}{n}
\end{equation}

We can conclude that every ciphertext is equally probable.
\section{RSA}
\subsection{(a)}
The public key $e$ can be select by $2 < e < (p - 1)(q - 1)$ and $e$ and $(p - 1)(q - 1)$ are coprime. So there are $\phi((p - 1)(q - 1))$ possible values for $e$.

Consider that $p = 101$ and $q = 113$, we have:
\begin{equation}
    \begin{aligned}
        \phi((p - 1)(q - 1)) & = \phi(100 \times 112)                      \\
                             & = \phi(2^2 \times 5^2  \times 2^4 \times 7) \\
                             & = \phi(2^6 \times 5^2 \times 7)             \\
                             & = \phi(2^6) \times \phi(5^2) \times \phi(7) \\
                             & = 2^5(2 - 1) \times 5(5 - 1) \times 6       \\
                             & = 3840                                      \\
    \end{aligned}
\end{equation}

So there are $3840$ possible values for $e$.
\subsection{(b)}
The ciphertext $c$ can be calculated by:
\begin{equation}
    \begin{aligned}
        c & = m^e \mod n             \\
          & = 9726^{3533} \mod 11413 \\
          & = 5761                   \\
    \end{aligned}
\end{equation}

So the ciphertext received by Bob is $5761$.

When Bob decrypts the ciphertext, he will do the following steps:
\begin{itemize}
    \item Calculate the private key.

          Firstly he can calculate $n = pq = 11413$ and then the private key $d$ by the following equation:
          \begin{equation}
              \begin{aligned}
                  d & = e^{-1} \mod (p - 1)(q - 1) \\
                    & = 3533^{-1} \mod 11200       \\
                    & = 6597                       \\
              \end{aligned}
          \end{equation}

    \item Calculate the plaintext $m$ by $m = c^d \mod n$. We have:
          \begin{equation}
              \begin{aligned}
                  m & = c^d \mod n             \\
                    & = 5761^{6597} \mod 11413 \\
                    & = 9726                   \\
              \end{aligned}
          \end{equation}
\end{itemize}

\subsection{(c)}
We know that $\Phi(n) = (p - 1)(q - 1)$, so if $\Phi(n)$ and n are known, we can calculate $p$ and $q$ by the following equation:
\begin{equation}
    \begin{cases}
        n       & = pq             \\
        \Phi(n) & = (p - 1)(q - 1) \\
    \end{cases}
\end{equation}

We can rewrite the equation as:
\begin{equation}
    \begin{cases}
        p + q = n - \Phi(n) + 1 \\
        pq = n                  \\
    \end{cases}
\end{equation}

Eliminate the variable q from the equations, we have:
\begin{equation}
    p^2 - (n - \Phi(n) + 1)p + n = 0
\end{equation}
which is a quadratic equation in the unknown $p$. And we can compute $p$ and $q$ in polynomial time by solving the above quadratic equation.

\section{Multi-Party Computation}
\subsection{(a)Paillier encryption}
\subsubsection{Encryption}
A simpler variant of the above key generation steps would be to set $g = n + 1$ and $\lambda = \Phi(n)$, which makes $\mu$ as follows:
\begin{equation}
    \begin{aligned}
        \mu & = (L(g^{\Phi(n)} \mod n^2))^{-1} \mod n                                                            \\
            & = (L((n + 1)^{\Phi(n)} \mod n^2))^{-1} \mod n                                                      \\
            & = (L(1 + \Phi(n)\cdot n + \sum\limits_{k = 2}^{\Phi(n)}\binom{\Phi(n)}{k}) \mod n^2)^{-1} \mod n \\
            & = (L((1 + \Phi(n)\cdot n) \mod n^2))^{-1} \mod n                                                 \\
    \end{aligned}
\end{equation}

As $1 + \Phi(n)\cdot n = 1 + pq(p - 1)(q - 1) < (pq)^2 = n^2$ and $L(x) = \frac{x - 1}{n}$, we can get:
\begin{equation}
    \begin{aligned}
        \mu & = (L((1 + \Phi(n)\cdot n) \mod n^2))^{-1} \mod n \\
            & = (\frac{1 + \Phi(n)\cdot n - 1}{n})^{-1} \mod n \\
            & = \Phi(n)^{-1} \mod n                            \\
    \end{aligned}
\end{equation}

So the public key is $(n, g) = (n, n + 1)$ and the private key is $(\lambda, \mu) = (\Phi(n), \Phi(n)^{-1} \mod n)$.

Substitute the given value $p$, $q$ and $r$, we can calculate $n$ as $p\cdot q = 11 * 17  = 187$, $g$ as $n + 1 = 188$ and $r = 83$. The ciphertext of $m = 175$ can be calculated by:
\begin{equation}
    \begin{aligned}
        c & = g^m \cdot r^n \mod n^2              \\
          & = 188^{175} \cdot 83^{187} \mod 187^2 \\
          & = 23911                               \\
    \end{aligned}
\end{equation}

\subsubsection{Homomorphic Addition of Paillier}
\begin{equation}
    \begin{aligned}
        \mathrm{Decrypt}((c_1 \cdot c_2) \mod n^2) & = \mathrm{Decrypt}(g^{m_1}r^{n}\cdot g^{m_2}r^{n} \mod n^2) \\
                                                   & = \mathrm{Decrypt}(g^{m_1 + m_2}(r^2)^n \mod n^2)
    \end{aligned}
\end{equation}

As $r$ is a random number, $r^2$ is also a random number. So we can get:
\begin{equation}
    \begin{aligned}
        \mathrm{Decrypt}((c_1 \cdot c_2) \mod n^2) & = \mathrm{Decrypt}(g^{m_1 + m_2}(r^2)^n \mod n^2) \\
                                                   & = m_1 + m_2                                       \\
    \end{aligned}
\end{equation}
\subsection{(b)Secret Sharing}
Firstly, we know for any bit $x, y$, $x \oplus x = 0$, $x \oplus 0 = x$ and $x \oplus y = y \oplus x$. So we can use the following algorithm to generate the shares:
\begin{equation}
    \begin{aligned}
        (a_1 \oplus a_2 \oplus a_3) & = (x_3 \oplus v) \oplus (x_1 \oplus v) \oplus (x_2 \oplus v) \\
                                    & = (x_1 \oplus x_2 \oplus x_3) \oplus (v \oplus v \oplus v)   \\
                                    & = 0 \oplus v                                                 \\
                                    & = v                                                          \\
    \end{aligned}
\end{equation}

So in order to compute $v_1 \oplus v_2$, we can compute $(a_1 \oplus a_2 \oplus a_3) \oplus (b_1 \oplus b_2 \oplus b_3)$ as follows:
\begin{equation}
    (a_1 \oplus a_2 \oplus a_3) \oplus (b_1 \oplus b_2 \oplus b_3) = v_1 \oplus v_2
\end{equation}

\section{Computational Security}
\subsection{(a)}
\paragraph*{Interchangeable}
"Interchangeable" means that if two objects are interchangeable, they can be substituted for each other in a scheme without compromising the security.
\paragraph*{Indistinguishable}
"Indistinguishable" means that an adversary cannot distinguish two different inputs or states from each other.
\paragraph*{Difference}
"Interchangeable" emphasizes the substitutablity of objects, while "indistinguishable" focuses the difficulty for an adversary to diffrenciate between these objects.
\subsection{(b)}
\subsubsection{Difinition}
A function $f(\lambda)$ is negligible if, for every polynomial function $p(\lambda)$, we have $\lim_{\lambda \to \infty}p(\lambda)f(\lambda) = 0$.
\subsubsection{Lemmas}
\paragraph*{Lemma 1}
Before all, we can prove that $\forall a > 1, b > 0$, $\frac{1}{a^{\lambda^b}}$ is negligible because give any polynomial function $p(\lambda) = a_n\lambda^n + a_{n - 1}\lambda^{n - 1} + \cdots + a_1\lambda + a_0$, there exists a function $\lambda^{n + 1}$ and $\lim_{\lambda \to \infty}\frac{p(\lambda)}{\lambda^{n + 1}} = 0$ because:
\begin{equation}
    \begin{aligned}
        \lim_{\lambda \to \infty}\frac{p(\lambda)}{\lambda^{n + 1}} & = \lim_{\lambda \to \infty}\frac{a_n\lambda^n + a_{n - 1}\lambda^{n - 1} + \cdots + a_1\lambda + a_0}{\lambda^{n + 1}}                                                                         \\
                                                                    & = \lim_{\lambda \to \infty}\frac{a_n\lambda^n}{\lambda^{n + 1}} + \frac{a_{n - 1}\lambda^{n - 1}}{\lambda^{n + 1}} + \cdots + \frac{a_1\lambda}{\lambda^{n + 1}} + \frac{a_0}{\lambda^{n + 1}} \\
                                                                    & = \lim_{\lambda \to \infty}\frac{a_n}{\lambda} + \frac{a_{n - 1}}{\lambda^2} + \cdots + \frac{a_1}{\lambda^{n + 1}} + \frac{a_0}{\lambda^{n + 1}}                                              \\
                                                                    & = 0                                                                                                                                                                                            \\
    \end{aligned}
\end{equation}

And also,$\forall a > 1, b > 0$, for $\frac{1}{a^{\lambda^b}}$ and let $\lambda^{'} = \lambda^b$, apply \textbf{Lópida's Law} and we have:
\begin{equation}
    \begin{aligned}
        0 \leq \lim_{\lambda \to \infty}\frac{\lambda^{n + 1}}{a^{\lambda^b}} & = \lim_{\lambda^{'} \to \infty} \frac{ \lambda^{'\frac{n+1}{b}}}{ a^{\lambda^{'}}}                                    \\
                                                                              & \leq \lim_{\lambda^{'} \to \infty} \frac{ \lambda^{'\lceil\frac{n+1}{b}\rceil}}{ a^{\lambda^{'}}}                     \\
                                                                              & = \lim_{\lambda^{'} \to \infty} \frac{\lceil\frac{n+1}{b}\rceil!}{a^{\lambda^{'}}(\ln a)^{\lceil\frac{n+1}{b}\rceil}} \\
                                                                              & = 0
    \end{aligned}
\end{equation}

So for any polynomial function $p(\lambda)$, we have:
\begin{equation}
    \begin{aligned}
        \lim_{\lambda \to \infty}p(\lambda)\frac{1}{a^{\lambda^b}} & = \lim_{\lambda \to \infty}\frac{p(\lambda)}{\lambda^{n + 1}}\cdot \lambda^{n + 1}\frac{1}{a^{\lambda^b}} \\
                                                                   & = 0                                                                                                       \\
    \end{aligned}
\end{equation}

\paragraph*{Lemma 2}
Also, we can prove that for all $g(\lambda)$, if $\lim_{\lambda \to \infty}g(\lambda) = \infty$, then $\frac{1}{\lambda^{g(\lambda)}}$ is negligible. Because give any polynomial function $p(\lambda) = a_n\lambda^n + a_{n - 1}\lambda^{n - 1} + \cdots + a_1\lambda + a_0$, there exists a function $\lambda^{n + 1}$ and $\lim_{\lambda \to \infty}\frac{p(\lambda)}{\lambda^{n + 1}} = 0$.

And also, for $g(\lambda)$ and $\frac{1}{\lambda^{g(\lambda)}}$, we have:
\begin{equation}
    \begin{aligned}
        \lim_{\lambda \to \infty}\lambda^{n + 1}\frac{1}{\lambda^{g(\lambda)}} & =\lim_{\lambda \to \infty}\frac{1}{\lambda^{g(\lambda) - n - 1}} \\
                                                                               & = 0                                                              \\
    \end{aligned}
\end{equation}

So for any polynomial function $p(\lambda)$, we have:
\begin{equation}
    \begin{aligned}
        \lim_{\lambda \to \infty}p(\lambda)\frac{1}{\lambda^{g(\lambda)}} & = \lim_{\lambda \to \infty}\frac{p(\lambda)}{\lambda^{n + 1}}\cdot \lambda^{n + 1}\frac{1}{\lambda^{g(\lambda)}} \\
                                                                          & = 0                                                                                                              \\
    \end{aligned}
\end{equation}

So $\frac{1}{\lambda^{g(\lambda)}}$ is negligible.

\paragraph*{Lemma 3}
Finally, we can prove that $\forall a > 0$, $\frac{1}{\lambda^a}$ is not negligible because give a polynomial function $p(\lambda) = \lambda^{a + 1}$, we have:
\begin{equation}
    \begin{aligned}
        \lim_{\lambda \to \infty}\lambda^{a + 1}\frac{1}{\lambda^a} & = \lim_{\lambda \to \infty}\frac{\lambda^{a + 1}}{\lambda^a} \\
                                                                    & = \lim_{\lambda \to \infty}\lambda                           \\
                                                                    & = \infty                                                     \\
    \end{aligned}
\end{equation}

\subsubsection{Prove}
\begin{itemize}
    \item $\frac{1}{2^{\lambda/2}}$ is negligible.
          \begin{equation}
              \frac{1}{2^{\lambda/2}} = \frac{1}{(\sqrt{2})^\lambda}
          \end{equation}
          As $\sqrt{2}$ is greater than $1$ and $1$ is greater than $0$ which corresponds to the case of \textbf{Lemma 1}, $\frac{1}{2^{\lambda/2}}$ is negligible.
    \item $\frac{1}{2^{\log(\lambda^2)}}$ is not negligible.
          \begin{equation}
              \begin{aligned}
                  \frac{1}{2^{\log(\lambda^2)}} & = \frac{1}{2^{2\log \lambda}} \\
                                                & = \frac{1}{4^{\log \lambda}}  \\
                                                & = \frac{1}{\lambda^{\log 4}}  \\
              \end{aligned}
          \end{equation}
          As $\log 4$ is greater than $0$ which corresponds to the case of \textbf{Lemma 3}, $\frac{1}{2^{\log(\lambda^2)}}$ is not negligible.
    \item $\frac{1}{\lambda^{\log \lambda}}$ is negligible.

          As $\lim_{\lambda \to \infty}\log \lambda = \infty$ which corresponds to the case of \textbf{Lemma 2}, $\frac{1}{\lambda^{\log \lambda}}$ is negligible.
    \item $\frac{1}{\lambda^2}$ is not negligible.

          As $2 > 0$ which corresponds to the case of \textbf{Lemma 3}, so $\frac{1}{\lambda^2}$ is not negligible.
    \item $\frac{1}{2^{(\log \lambda)^2}}$ is negligible.
          \begin{equation}
              \begin{aligned}
                  \frac{1}{2^{(\log \lambda)^2}} & = \frac{1}{2^{\log \lambda \cdot \log \lambda}} \\
                                                 & = \frac{1}{\lambda^{\log 2\cdot \log \lambda}}  \\
              \end{aligned}
          \end{equation}
          As $\lim_{\lambda \to \infty} \log 2\cdot \log \lambda = \infty$ which corresponds to the case of \textbf{Lemma 2}, $\frac{1}{2^{(\log \lambda)^2}}$ is negligible.
    \item $\frac{1}{(\log \lambda)^2}$ is not negligible.

          Select $p(\lambda) = (\log \lambda)^2$, we have:
          \begin{equation}
              \begin{aligned}
                  \lim_{\lambda \to \infty}\lambda^2\cdot \frac{1}{(\log \lambda)^2} & =  \lim_{\lambda \to \infty}\frac{\lambda^2}{(\log \lambda)^2} \\
                                                                                     & = \lim_{\lambda \to \infty}\frac{2\lambda^2}{2\log \lambda}    \\
                                                                                     & = \lim_{\lambda \to \infty} 2\lambda^2                         \\
                                                                                     & = \infty                                                       \\
              \end{aligned}
          \end{equation}
          So $\frac{1}{(\log \lambda)^2}$ is not negligible.
    \item $\frac{1}{\lambda^{1/\lambda}}$ is not negligible.

          Select $p(\lambda) = \lambda$, we have:
          \begin{equation}
              \begin{aligned}
                  \lim_{\lambda \to \infty}\lambda\cdot \frac{1}{\lambda^{1/\lambda}} & = \lim_{\lambda \to \infty}\frac{\lambda}{\lambda^{1/\lambda}} \\
                                                                                      & = \lim_{\lambda \to \infty}\lambda^{1 - 1/\lambda}             \\
                                                                                      & = \infty                                                       \\
              \end{aligned}
          \end{equation}
          So $\frac{1}{\lambda^{1/\lambda}}$ is not negligible.
    \item $\frac{1}{\sqrt{\lambda}}$ is not negligible.

          As $\frac{1}{2}$ is greater than $0$ which corresponds to the case of \textbf{Lemma 3}, $\frac{1}{\sqrt{\lambda}}$ is not negligible.
    \item $\frac{1}{2^{\sqrt{\lambda}}}$ is negligible.

          As $2$ is greater than $1$ and $\frac{1}{2}$ is greater than $0$ which corresponds to the case of \textbf{Lemma 1}, $\frac{1}{2^{\sqrt{\lambda}}}$ is negligible.
\end{itemize}
\subsection{(c)}
\subsubsection{$f + g$}
Since $f$ and $g$ are negligible, we have for any polynomial function $p_1(\lambda)$ and $p_2(\lambda)$:
\begin{equation}
    \begin{aligned}
        \lim_{\lambda \to \infty}p_1(\lambda)\cdot f(\lambda) & = 0 \\
        \lim_{\lambda \to \infty}p_2(\lambda)\cdot g(\lambda) & = 0 \\
    \end{aligned}
\end{equation}

For any polynomial function $p(\lambda)$, select $p_1(\lambda) = p(\lambda)$ and $p_2(\lambda) = p(\lambda)$, we have:
\begin{equation}
    \begin{aligned}
        \lim_{\lambda \to \infty}p(\lambda)\cdot (f(\lambda) + g(\lambda)) & = \lim_{\lambda \to \infty}p(\lambda)\cdot f(\lambda) + p(\lambda)\cdot g(\lambda) \\
                                                                           & = 0 + 0                                                                            \\
                                                                           & = 0                                                                                \\
    \end{aligned}
\end{equation}

So $f + g$ is negligible.
\subsubsection{$f\cdot g$}
For any polynomial function $p(\lambda)$, select $p_1(\lambda) = p(\lambda)$ and $p_2(\lambda) = 1$, we have:
\begin{equation}
    \begin{aligned}
        \lim_{\lambda \to \infty}p(\lambda)\cdot (f(\lambda)\cdot g(\lambda)) & = \lim_{\lambda \to \infty}(p(\lambda)\cdot f(\lambda))\cdot (1 \cdot g(\lambda)) \\
                                                                              & = 0\cdot 0                                                                        \\
                                                                              & = 0                                                                               \\
    \end{aligned}
\end{equation}
\subsubsection{$f / g$}
For example, select $f(\lambda) = \frac{1}{2^\lambda}$ and $g(\lambda) = \frac{1}{4^\lambda}$. Obviously, $f(\lambda)$ and $g(\lambda)$ are negligible. But $f(\lambda) / g(\lambda) = 2^\lambda$ is surely not negligible.

\end{document}
